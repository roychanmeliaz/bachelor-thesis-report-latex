\begin{center}
  \large\textbf{ABSTRACT}
\end{center}

\addcontentsline{toc}{chapter}{ABSTRACT}

\vspace{2ex}

\begingroup
  % Menghilangkan padding
  \setlength{\tabcolsep}{0pt}

  \noindent
  \begin{tabularx}{\textwidth}{l >{\centering}m{3em} X}
    % Ubah kalimat berikut dengan nama mahasiswa
    \emph{Name}     &:& Muhammad Roychan Meliaz \\

    % Ubah kalimat berikut dengan judul tugas akhir dalam Bahasa Inggris
    \emph{Title}    &:& ITS \textit{Autonomous Car} Maneuver at Roundabouts or U-Turn using \textit{Deep Reinforcement Learning} \\

    % Ubah kalimat-kalimat berikut dengan nama-nama dosen pembimbing
    \emph{Advisors} &:& 1. Prof. Dr. Ir. Mauridhi Hery Purnomo, M.Eng. \\
                    & & 2. Dr. I Ketut Eddy Purnama, ST., MT. \\
  \end{tabularx}
\endgroup

% Ubah paragraf berikut dengan abstrak dari tugas akhir dalam Bahasa Inggris
\emph{An autonomous car or autonomous vehicle is a vehicle
	which has the ability to drive independently as if controlled by humans using a series of artificial intelligence. In this study, we propose research on the development of an autonomous vehicle iCar ITS (Intelligent Car Institut Teknologi Sepuluh Nopember) by developing a maneuvering system for autonomous vehicles at roundabouts or u-turns in a simulated environment. In the simulation environment, the model used is a vehicle model adapted to iCar. The development of autonomous vehicle navigation and maneuvering systems is carried out using the Deep Reinforcement Learning method, a branch of Machine Learning. From this research, the results obtained are reinforcement learning models that are able to maneuver at roundabouts with intersections and roundabouts without intersections with a mean deviation value of the angle of the path of 27,011° and 30,068°, respectively, able to maneuver without collision for an average of 13.3 seconds and 7.9 seconds, and with an average speed of 27.0 kmph and 28.5 kmph.}
	
	\iffalse
	it is hoped that reinforcement learning algorithms on autonomous cars can be produced that are able to maneuver at roundabouts or u-turns efficiently.
	\fi

% Ubah kata-kata berikut dengan kata kunci dari tugas akhir dalam Bahasa Inggris
\emph{Keywords}: \emph{Autonomous Vehicle}, \emph{Reinforcement Learning}, \emph{Deep Learning}, \emph{Simulation}.
