\begin{center}
  \large\textbf{ABSTRAK}
\end{center}

\addcontentsline{toc}{chapter}{ABSTRAK}

\vspace{2ex}

\begingroup
  % Menghilangkan padding
  \setlength{\tabcolsep}{0pt}

  \noindent
  \begin{tabularx}{\textwidth}{l >{\centering}m{2em} X}
    % Ubah kalimat berikut dengan nama mahasiswa
    Nama Mahasiswa    &:& Muhammad Roychan Meliaz \\

    % Ubah kalimat berikut dengan judul tugas akhir
    Judul Tugas Akhir &:&	Manuver \textit{Autonomous Car} ITS di Bundaran atau \textit{U-Turn} Menggunakan \textit{Deep Reinforcement Learning} \\

    % Ubah kalimat-kalimat berikut dengan nama-nama dosen pembimbing
    Pembimbing        &:& 1. Prof. Dr. Ir. Mauridhi Hery Purnomo, M.Eng. \\
                      & & 2. Dr. I Ketut Eddy Purnama, ST., MT. \\
  \end{tabularx}
\endgroup

% Ubah paragraf berikut dengan abstrak dari tugas akhir
\textit{Autonomus Car }atau kendaraan otonom merupakan kendaraan yang memiliki kemampuan untuk berkendara secara mandiri layaknya dikendalikan manusia dengan mengunakan rangkaian kecerdasan buatan. Pada penelitian ini kami mengajukan riset pengembangan kendaraan otonom iCar ITS (\textit{Intelligent Car }Institut Teknologi Sepuluh Nopember) dengan mengembangkan sistem manuver kendaraan otonom di bundaran atau u-turn dalam lingkungan yang disimulasikan. Dalan lingkungan simulasi, model yang digunakan adalah model kendaraan yang disesuaikan dengan iCar. Pengembangan sistem navigasi dan manuver kendaraan otonom dilakukan menggunakan metode \textit{Deep Reinforcement Learning}, salah satu cabang dari \textit{Machine Learning}. Pada penelitian ini, didapatkan hasil model reinforcement learning yang mampu melakukan manuver bundaran simpang empat dan bundaran tanpa simpang dengan nilai rerata deviasi sudut dari jalurnya masing-masing senilai 27.011° dan 30.068°, mampu bermanuver tanpa \textit{collision} selama rerata 13.3 detik dan 7.9 detik, serta dengan kecepatan rerata 27.0 kmpj dan 28.5 kmpj.

\iffalse
Dari penelitian ini diharapkan dapat dihasilkannya algoritma \textit{reinforcement learning }pada mobil otonom yang mampu melakukan manuver di bundaran atau u-turn dengan efisien.
\fi

% Ubah kata-kata berikut dengan kata kunci dari tugas akhir
Kata Kunci: Kendaraan otonom, \textit{Reinforcement Learning}, \textit{Deep Learning}, Simulasi.
