\chapter{PENUTUP}
\label{chap:penutup}

% Ubah bagian-bagian berikut dengan isi dari penutup

\section{Kesimpulan}
\label{sec:kesimpulan}

Dari perancangan dan pengujian sistem yang telah dilakukan, diperoleh beberapa kesimpulan sebagai berikut. Kendala dan kekurangan yang penulis hadapi juga penulis tuliskan pada bagian saran dengan harapan untuk membantu pengembangan penelitian selanjutnya.

\begin{enumerate}[nolistsep]

  \item Sistem perencanaan gerakan mobil otonom berbasis algoritma DQN yang dirancang mampu melakukan akselerasi dan \textit{steer} yang mampu mengontrol mobil otonom dan memahami kondisi sekitarnya.

  \item Penggunaan \textit{state } citra dengan segmentasi lanjutan; yaitu menghasilkan output citra \textit{drivable }dan \textit{non-drivable} menghasilkan hasil yang lebih baik daripada segmentasi biasa dengan performa \textit{average\_episode\_length} 42.8\% lebih baik, \textit{average\_speed} 15.3\% lebih baik, dan \textit{average\_distance\_difference} 112.4\% lebih baik.

\end{enumerate}

\section{Saran}
\label{chap:saran}

Untuk pengembangan selanjutnya pada topik penelitian perencanaan gerakan mobil otonom dengan menggunakan algoritma DQN, terdapat beberapa saran yang diberikan, antara lain sebagai berikut:

\begin{enumerate}[nolistsep]

  \item Sensor yang digunakan untuk navigasi dari \textit{agent }menggunakan algoritma DQN dapat ditambah menjadi lebih banyak seperti GPS serta LIDAR, agar \textit{agent} dapat bertindak dengan parameter yang lebih lengkap.

  \item Proses training dapat dilakukan dengan waktu yang lebih panjang serta menggunakan mesin dengan tenaga komputasi yang lebih kuat agar didapatkan model yang konvergen.

  \iffalse
  \item Memperbaiki sistem award, dimana sebaiknya memikirkan sudut kendaraan terhadap pusat bundaran, serta jarak kendaraan terhadap bagian tengah jalan.

  \item Membuat tolak ukur metrik yang memadai untuk bagian pengujian model berupa plot riwayat gerakan kendaraan serta plot kecepatan kendaraan.
  \fi

\end{enumerate}
