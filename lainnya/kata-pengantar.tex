\begin{center}
  \Large
  \textbf{KATA PENGANTAR}
\end{center}

\addcontentsline{toc}{chapter}{KATA PENGANTAR}

\vspace{2ex}

% Ubah paragraf-paragraf berikut dengan isi dari kata pengantar

Puji dan syukur kehadirat Allah SWT atas segala limpahan berkah, rahmat, serta ridho-Nya, penulis dapat menyelesaikan penelintian ini dengan judul \textbf{Manuver \textit{Autonomous Car }ITS di Bundaran atau \textit{U-Turn }Menggunakan \textit{Deep Reinforcement Learning}.}

Penelitian ini disusun dalam rangka pemenuhan bidang riset di Departemen Teknik Komputer ITS, sera digunakan sebagai persyaratan menyelesaikan pendidikan Sarjana. Penelitian ini dapat diselesaikan tidak lepas dari bantuan berbagai pihak. Oleh karena itu, penulis mengucapkan terimakasih kepada:

\begin{enumerate}[nolistsep]
	
	\item Keluarga, Ibu, Bapak dan Saudara tercinta yang telah memberikan dorongan baik secara spiritual dan material dalam penyelesaian buku penelitian ini. 
	
	\item Bapak Supeno Mardi Susiki Nugroho, selaku Kepala Departemen Teknik Komputer, Fakultas Teknologi Elektro dan Informatika Cerdas, Institut Teknologi Sepuluh Nopember.
	
	\item Bapak Mauridhi Hery Purnomo selaku dosen pembimbing I dan Bapak I Ketut Eddy Purnama selaku dosen pembimbing II yang memberikan arahan selama mengerjakan Tugas Akhir ini.

	\item Bapak Muhtadin yang telah menawarkan judul Tugas Akhir dan memberikan arahan dan masukan selama mengerjakan Tugas Akhir ini.

	\item Bapak-ibu dosen pengajar Departemen Teknik Komputer atas pengajaran, bimbingan, serta perhatian yang diberikan kepada penulis selama ini.
	
	\item Sahabat Rumah Anak TK yang selalu ada dan bersedia membantu dan menyemangati penulis tidak hanya dalam pengerjaan Tugas Akhir, namun juga selama masa perkuliahan.
	
	\item Seluruh teman-teman angkatan e57, khususnya Teknik Komputer ITS angkatan 2017 yang sudah bersama penulis mengisi masa perkuliahan dengan sukacita.
	
\end{enumerate}

Penulis menyadari sepenuhnya bahwa Tugas Akhir ini masih jauh dari kesempurnaan. Untuk itu, penulis sangat terbuka untuk kritik dan saran yang bersifat membangun. Semoga penelitian ini dapat memberikan manfaat bagi kita semua. Amin.

\begin{flushright}
  \begin{tabular}[b]{c}
    % Ubah kalimat berikut dengan tempat, bulan, dan tahun penulisan
    Surabaya, Juni 2021\\
    \\
    \\
    \\
    \\
    % Ubah kalimat berikut dengan nama mahasiswa
    Muhammad Roychan Meliaz
  \end{tabular}
\end{flushright}
