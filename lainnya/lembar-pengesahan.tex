\begin{center}
	\large
  \textbf{LEMBAR PENGESAHAN}
\end{center}

% Menyembunyikan nomor halaman
\thispagestyle{empty}

\begin{center}
  % Ubah kalimat berikut dengan judul tugas akhir
  \textbf{Manuver Autonomous Car ITS di Bundaran atau U-Turn Menggunakan Deep Reinforcement Learning}
\end{center}

\begingroup
  % Pemilihan font ukuran small
  \small

  \begin{center}
    % Ubah kalimat berikut dengan pernyataan untuk lembar pengesahan
    Tugas Akhir ini disusun untuk memenuhi salah satu syarat memperoleh gelar Sarjana Teknik di Institut Teknologi Sepuluh Nopember Surabaya
  \end{center}

  \begin{center}
    % Ubah kalimat berikut dengan nama dan NRP mahasiswa
    Oleh: Muhammad Roychan Meliaz(NRP. 0721 17 4000 0012)
  \end{center}

  \begin{center}
    % Ubah kalimat-kalimat berikut dengan tanggal ujian dan periode wisuda
    Tanggal Ujian : Juli 2021\\
    Periode Wisuda : September 2021
  \end{center}

  \begin{center}
    Disetujui Oleh:
  \end{center}

  \begingroup
    % Menghilangkan padding
    \setlength{\tabcolsep}{0pt}

    \noindent
    \begin{tabularx}{\textwidth}{X c}
      % Ubah kalimat-kalimat berikut dengan nama dan NIP dosen pembimbing pertama
      Prof. Dr. Ir. Mauridhi Hery Purnomo, M.Eng.          & (Pembimbing I) \\
      NIP: 	19580916 198601 1 001       & ................................... \\
      &  \\
      &  \\
      % Ubah kalimat-kalimat berikut dengan nama dan NIP dosen pembimbing kedua
      Dr. I Ketut Eddy Purnama, ST., MT.     & (Pembimbing II) \\
      NIP: 19690730 199512 1 001        & ................................... \\
      &  \\
      &  \\
      % Ubah kalimat-kalimat berikut dengan nama dan NIP dosen penguji pertama
      %Dr. Galileo Galilei, S.T., M.Sc.  & (Penguji I) \\
      %NIP: 15640215 164201 1 001        & ................................... \\
      &  \\
      &  \\
      % Ubah kalimat-kalimat berikut dengan nama dan NIP dosen penguji kedua
      %Friedrich Nietzsche, S.T., M.Sc.  & (Penguji II) \\
      %NIP: 18441015 190008 1 001        & ................................... \\
      &  \\
      &  \\
      % Ubah kalimat-kalimat berikut dengan nama dan NIP dosen penguji ketiga
      %Alan Turing, ST., MT.             & (Penguji III) \\
      %NIP: 19120623 195406 1 001        & ................................... \\
    \end{tabularx}
  \endgroup

  \vspace{2ex}

  \begin{center}
    % Ubah kalimat berikut dengan jabatan kepala departemen
    Mengetahui, \\
    Kepala Departemen Teknik Komputer FTEIC ITS\\

    \vspace{8ex}

    % Ubah kalimat-kalimat berikut dengan nama dan NIP kepala departemen
    \underline{Dr. Supeno Mardi Susiki Nugroho, ST., MT.} \\
    NIP. 19700313 199512 1 001
  \end{center}
\endgroup
